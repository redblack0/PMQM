\documentclass{f4_beamer}

\title{"Earned value" Methode aus dem klassischen Projektmanagement}
%\subtitle{Untertitel}
\author{Nick Marlon Grunert}
\date{\today}

\begin{document}

\section{Einführung}

\begin{frame}{Einführung}
    \begin{itemize}
        \item Einführungstext
        \item Noch mehr
        \item bitte funktioniere
    \end{itemize}
\end{frame}

\section{simpleSlides}

% [fragile] is needed for the \verb|...| command to work. It is magic and show
% it up if you want to know how it works.
% NOTE: keep in mind that the use of fragile the build time increases
\begin{frame}[fragile]
    Change the labels using \verb|\item[label text]| in an
    \texttt{itemize} environment

    \begin{itemize}
        \item This is my first point
        \item Another point I want to make
        \item[!] A point to exclaim something!
        \item[$\blacksquare$] Make the point fair and square.
        \item[NOTE] This entry has no bullet
        \item[] A blank label?
    \end{itemize}
\end{frame}

\begin{frame}[fragile]
    Change the labels using \verb|\item[label text]| in an
    \texttt{enumerate} environment
    \begin{enumerate}
        \item This is my first point
        \item Another point I want to make
        \item[!] A point to exclaim something!
        \item[$\blacksquare$] Make the point fair and square.
        \item[NOTE] This entry has no bullet
        \item[] A blank label?
    \end{enumerate}
\end{frame}
\section{pictureSlides}

\begin{frame}
    \begin{figure}[htp]
        \centering
        \includegraphics[width=.8\textwidth, keepaspectratio]{%
            % image has to be relative to the actual f4_beamer.tex because
            % this file is being included there. Hence, if the image is relative
            % to this current directory it will not work.
            pictureSlides/test-image.jpg}
        \caption{%
            https://www.pexels.com/de-de/foto/natur-dunkel-wald-baume-6992/
        }
    \end{figure}
\end{frame}

\begin{frame}
    \begin{figure}[!tbp]
        \centering
        % https://www.namsu.de/Extra/befehle/Minipage.html
        \begin{minipage}{0.5\textwidth}
            \includegraphics[width=\textwidth]{pictureSlides/test-image.jpg}
            \caption{picture number 1}
        \end{minipage}
        \hfill
        \begin{minipage}{0.4\textwidth}
            \includegraphics[width=\textwidth]{pictureSlides/test-image.jpg}
            \caption{picture number 2}
        \end{minipage}
        \caption{this is very interesting}
    \end{figure}
\end{frame}

\begin{frame}
    \begin{figure}[!tbp]
        \centering
        \begin{minipage}{0.4\textwidth}
            \includegraphics[width=\textwidth]{pictureSlides/test-image.jpg}
            \caption{picture number 1}
        \end{minipage}
        \hfill
        \begin{minipage}{0.5\textwidth}
            \lipsum[1][1-5]
        \end{minipage}
    \end{figure}
\end{frame}



\section{Prognosen}
\begin{frame}[fragile]
    Mit Hilfe der berechneten Werte können Prognosen für den weiteren Projektverlauf erstellt werden
    \begin{itemize}
        \item Viele Rechnungsmethode sind dabei denkbar
    \end{itemize}
    Regel: Überschrittenes Budget kann tendenziell nicht wieder ausgeglichen werden
\end{frame}


\section{Software Development}
\begin{frame}[fragile]
    Was muss getan werden um "earned value" auf Softwareentwicklung anzuwenden?
    \begin{itemize}
        \item Wochen als Iterationen interpretieren
        \item Kosten der Arbeitspakete aufsummieren
    \end{itemize}
\end{frame}
\begin{frame}[fragile]
    Problem - Arbeitspakete sind meist parallel
    \begin{itemize}
        \item Wie gut kann die Methode damit umgehen?
        \item Inwieweit sind Pakete halb fertig?
    \end{itemize}
\end{frame}


\end{document}
